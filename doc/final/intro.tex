\chapter{Introduction}
\label{chap:intro}

Information-Centric Networking (ICN) is a field of research which advocates for 
a new communication paradigm for the Internet: the departure from 
its host-centric model and the adoption of a content-centric model, enabling 
the direct addressing of content, independently of its 
location. This position is motivated by increasing primary use of the Internet 
for content dissemination, rather than for explicit communication between 
hosts~\cite{Xylomenos2013}.\vertbreak 

This work was prepared as an `hands-on' introduction to 
the field of ICN by taking advantage of the software packages offered by 
Project CCNx~\cite{website:ccnx}, an implementation of the Content Centric 
Networking (CCN) architecture~\cite{Jacobson2009}. We 
run a series of experiments using a simple and exemplifying testbed, evaluating 
the performance of the CCNx implementation over several network parameters and 
its suitability for use in constrained devices.

\section{Content Centric Networking}
\label{sec:intro-ccnx}

Content-Centric Networking (CCN)~\cite{Jacobson2009} presents itself as a 
novel networking paradigm which moves away from the host-centric communication 
model: instead of retrieving content by first determining its location, CCN 
proposes accessing content directly, independently of its location in the 
network, by using content names as addresses.\vertbreak

Communication in CCNs is similar to a publish\slash subscribe model, in the 
sense that it is also driven by 
the consumers of data --- the subscribers --- which release Interest packets 
into the network, eventually received by other CCN nodes. These packets 
announce a subscriber's desire to fetch particular content via the specification 
of a content name field. Holders of content which matches the content name 
--- the publishers --- respond with Data packets, also referred to as Content 
Objects.

\section{Objectives}

The main objectives of this work are the following:\vertbreak

\textbf{1)} Acquire detailed knowledge about a particular ICN approach, by 
            planning and deploying a simple ICN testbed based on 
            Project CCNx~\cite{website:ccnx}, an implementation of the Content 
            Centric Networking (CCN) approach~\cite{Jacobson2009}.\vertbreak

\textbf{2)} Validate some of the benefits claimed by ICN approaches and in some 
            cases compare the network performance of CCNx against equivalent 
            `host centric' solutions. The 
            measurement parameters may vary with each test. Since the 
            objective of CCNs is to reduce latency experienced by clients and 
            network traffic load towards the origins of 
            content~\cite{Jacobson2009}, most of the evaluations are made by 
            comparing values of throughput, latency and network traffic 
            load.\vertbreak

\textbf{3)} We have the clear intention to test the performance of CCNx on 
            `real-world' constrained devices, differing our work from other 
            studies (of different nature), performed in the near-past by Vahlenkamp et 
            al.~\cite{Wahlisch2012, Vahlenkamp2012}. The idea is to evaluate 
            how suitable is the CCNx implementation for direct application 
            in constrained devices such those applied to e.g. Wireless Sensor 
            Networks (WSNs) or Vehicular Networks (VANETS)~\cite{Amadeo2013,Grassi2013}.

