\chapter{CCNx Quick Reference}
\label{app:ccnx-specificss}

Here we provide a quick reference for several CCNx commands and applications 
used in this work. We follow a similar section structure as that of 
Section~\ref{subsec:ccnx-specifics}.

\section{CCNx Forwarding Tables}

To add a route allowing a node \verb+ccnx1+ to forward Interest packets 
directed at content domain \verb+ccnx:/content/files/+, the following 
\verb+ccndc+ command shall be used:

\begin{verbatim}
user@ccnx1:-# ccndc add ccnx:/content/files/ udp 172.16.1.101
\end{verbatim}

This command will tell \verb+ccnd+ to forward any Interests for content matching 
the prefix \verb+ccnx:/content/files/+ to the IP address 172.16.1.101, over UDP.

\section{Disseminating Content with CCNx}

\cprotect\subsection{Usage of \verb+ccnr+ and \verb+ccngetfile+}

In CCNx, content sources use a specific repository application for storing 
content and make it available in CCNx networks by responding to 
matching Interests, \verb+ccnr+~\cite{website:ccnx-commands}. The following 
sequence of commands initializes a repository and inserts a file into it, 
making it addressable via the name \verb+ccnx:/CCN_REPO_DIR/file1.dat+:

\begin{verbatim}
user@cs:-# ccnr
user@cs:-# ccnputfile ccnx:/CCN_REPO_DIR/file1.dat file1.dat
\end{verbatim}

The 
\verb+ccngetfile+ application~\cite{website:ccnx-commands} 
can then be used to release Interest packets to the CCNx network, querying 
for particular content, e.g. (assuming that 
\verb+CCN_REPO_DIR+ = \verb+/content/files+):

\begin{verbatim}
user@ccnx1:-# ccngetfile ccnx:/content/files/file1.dat file1.dat.local
\end{verbatim}

The command above releases Interests for the content 
\verb+ccnx:/content/files/file1.dat+ and saves the file locally as 
\verb+file1.dat.local+. Being a CCNx application, it follows CCN's `1 Data packet 
per Interest' principle~\cite{Jacobson2009}, but with pipelining of Interests, 
i.e. the application immediately releases $N$ Interest packets into the network 
before the actual reception of Data packets. The actual behavior of this 
exchange (flow control) is analyzed in some of the tests specified in 
Section~\ref{sec:protocol}.

\cprotect\subsection{Usage of \verb+ccnsendchunks+ and \verb+ccncatchunks2+}

We provide examples of the use 
of both commands below, for an exchange of content 
\verb+ccnx:/content/files/file1.dat+ between nodes \verb+ccnx1+ (sender) 
and \verb+ccnx2+ (receiver):

\begin{verbatim}

user@ccnx1:-# ccnsendchunks -b 4096 ccnx:/content/files/file1.dat < file1.dat.ccnx1

(...)

user@ccnx2:-# ccncatchunks2 -p 31 ccnx:/content/files/file1.dat > file1.dat.ccnx2

\end{verbatim}

With the combination of commands shown above, \verb+ccnx1+ gets ready to 
send Data packets, each one containing blocks of 4096 bytes of the 
file \verb+file1.dat.ccnx1+. This content may be addressed via 
the content name \verb+ccnx:/content/files/file1.dat+. On the other end, 
\verb+ccnx2+ starts sending batches of (at most) 31 Interest packets for the 
content \verb+ccnx:/content/files/file1.dat+, saving the file as 
\verb+file1.dat.ccnx2+ when all the chunks are finally transmitted.

\cprotect\subsection{Usage of VLC plugin}

E.g. for reproducing some content addressed 
as \verb+ccnx:/content/files/video.avi+, located somewhere in the 
networks, VLC should be called in the following way (note the triple slash `\slash'):

\begin{verbatim}

user@ccnx1:-# vlc ccnx:///content/files/video.avi

\end{verbatim}
